\documentclass[11pt,a4paper]{article}
\usepackage{ngerman}
\usepackage[ngerman]{babel}
\usepackage[utf8x]{inputenc}
\usepackage[T1]{fontenc}
\usepackage{lmodern}
\usepackage{marvosym}
\usepackage{amsfonts,amsmath,amssymb}
\usepackage{textcomp}
\usepackage{pifont}
\usepackage{ifpdf}
\usepackage{enumitem}
\usepackage{paralist}
\usepackage{mdwlist}
\usepackage{fancybox}
\usepackage[pdftex]{color}
\ifpdf
  \usepackage[pdftex]{graphicx}
\else
  \usepackage[dvips]{graphicx}\fi

\usepackage{scrextend} % Paket notwendig um Einzug bei Fußnoten zu unterbinden

\pagestyle{empty}

\usepackage[scale=0.775]{geometry}
\setlength{\parindent}{0pt}
\addtolength{\parskip}{6pt}

\def\firstname{Pascal}
\def\familyname{Bernhard}
\def\FileAuthor{\firstname~\familyname}
\def\FileTitle{\firstname~\familyname Forschungsbericht Eurasische Union}
\def\FileSubject{Forschungsbericht}
\def\FileKeyWords{\firstname~\familyname, Eurasische Union, Russland}

\renewcommand{\ttdefault}{pcr}
\hyphenation{ins-be-son-de-re}
\usepackage{url}
\urlstyle{tt}
\ifpdf
  \usepackage[pdftex,pdfborder=0,breaklinks,baseurl=http://,pdfpagemode=None,pdfstartview=XYZ,pdfstartpage=1]{hyperref}
  \hypersetup{
    pdfauthor   = \FileAuthor,%
    pdftitle    = \FileTitle,%
    pdfsubject  = \FileSubject,%
    pdfkeywords = \FileKeyWords,%
    pdfcreator  = \LaTeX,%
    pdfproducer = \LaTeX}
\else
  \usepackage[dvips]{hyperref}
\fi

% Farben werden hier definiert
\definecolor{yellowcolor}{RGB}{210,160,0}
\definecolor{firstnamecolor}{RGB}{56,115,179}
\definecolor{bluecolor}{RGB}{56,115,179}
\definecolor{purplecolor}{RGB}{156,15,86}
\hypersetup{pdfborder=0 0 0}

% Gleiche Schriftart für Hyperlinks
\urlstyle{same}


%  Gefrickel um URL-Links vernünftig umzubrechen
\makeatletter
\g@addto@macro\UrlBreaks{
  \do\a\do\b\do\c\do\d\do\e\do\f\do\g\do\h\do\i\do\j
  \do\k\do\l\do\m\do\n\do\o\do\p\do\q\do\r\do\s\do\t
  \do\u\do\v\do\w\do\x\do\y\do\z\do\&\do\1\do\2\do\3
  \do\4\do\5\do\6\do\7\do\8\do\9\do\0}
% \def\do@url@hyp{\do\-}

% Hiermit soll einer übervolle Box verhindert werden -- funktioniert sogar irgendwie
\g@addto@macro\UrlSpecials{\do\/{\mbox{\UrlFont/}\hskip 0pt plus 1pt}}
\makeatother

% Kein Einzug bei Fußnoten
\deffootnote[2em]{2em}{1em}{\textsuperscript{\thefootnotemark}\,}

% Serifenlose Schrift für das gesamte Dokument
\renewcommand*\familydefault{\sfdefault}


\begin{document}
\sffamily   % for use with a résumé using sans serif fonts;
%\rmfamily  % for use with a résumé using serif fonts;
\hfill%
\begin{minipage}[t]{.6\textwidth}
\raggedleft%
	{\bfseries {\color{firstnamecolor}\firstname}~{\color{bluecolor}\familyname}}\\[.35ex]
	\small\itshape%
	\textbf{Diplompolitologe \& Master of Business Administration}\\
	Schwalbacher Straße 7\\
	12161 Berlin\\[.35ex]
	\Mobilefone~+49 162 32 39 557 \\
	\Letter~\href{mailto:pbcoaching@rppr.de}{pbcoaching@rppr.de}

\end{minipage}\\[0.5em]
%
{\color{firstnamecolor}\rule{\textwidth}{.25ex}}
%
\begin{minipage}[t]{.4\textwidth}
	\raggedright%
	% {\bfseries {\color{firstnamecolor}
	\vspace*{1em}
	\textbf{\color{bluecolor} {\Large Antrag für Forschungsbericht zu Regionalorganisationen im postsowjetischen Raum}} \\
	%\\[.35ex]
	% }}
	\small%

\end{minipage}
%
\hfill
%
\begin{minipage}[t]{.4\textwidth}
	\raggedleft % US style
%	\today
	%April 6, 2006 % US informal style
	%05/04/2006 % UK formal style
\end{minipage}\\[2.2em]



\subsection*{\color{yellowcolor} Die Eurasische Union angesichts Russlands wirtschaftlicher Lage}


Die Eurasische Wirtschaftsunion (EAWU) ist der neueste Versuch Russlands, die Ex-Republiken der zerfallenen Sowjetunion unter seiner Führung zu integrieren. Bereits 1993 hatten 12 Mitglieder der Gemeinschaft Unabhängiger Staaten einen Vertrag über eine Wirtschaftsunion geschlossen. Dieser verfehlte das gesetzte Ziel einer Freihandelszone samt Zollunion und einer gemeinsamen Währung ebenso wie die 1995 zwischen Russland, Belarus und Kasachstan vereinbarte Zollunion, die im Jahre 2000 zur Eurasischen Wirtschaftsgemeinschaft erweitert wurde. Die Ergebnisse der ersten Integrationsprojekte mögen nicht weiter verwundern angesichts der zunehmend konfrontativen Außenpolitik Russlands unter Präsident Vladimir Putin gegenüber seinen Nachbarn, die explizit die Wiederherstellung des Großmachtstatus zum Ziel hat. Zahlreiche Beobachter sehen auch die Eurasische Wirtschaftsunion als weiteren Versuch Moskaus seine imperialen Ambitionen zu verfolgen, denn auch dieses Projekt wird von der Russischen Föderation als größte Wirtschaft und politisches Schwergewicht eindeutig dominiert (vgl. Blank 2013, Mankoff 2013 und Sagorskij 2014).


\paragraph{Russische Außenpolitik}
Mehrere Autoren amerikanischer Institutionen sehen Vladimir Putins Außenpolitik von einer klassischen neo-realistischen Weltsicht bestimmt, die eine Stärkung von Russlands Macht und Position unter den Nationen als vorrangiges Ziel hat und internationale Beziehungen als Nullsummenspiel betrachtet (vgl. Donaldson et al. 2014, Gvosdev 2014, Mankoff 2009). Im konfliktreichen Verlauf russischer Außenpolitik in den letzten Jahren, vor allem gegenüber seinen unmittelbaren Nachbarn, gibt es zahlreiche Belege für eine neo-realistische Interpretation zu Motiven und Zielsetzung Moskaus. An dieser Stelle seien stellvertretend nur einige Ereignisse aus der jüngsten Vergangenheit genannt: Russland-Georgien Krieg, Gesetz über den Einsatz von Streitkräften zum Schutz russischer Staatsbürger im Ausland, Annexion des Halbinsel Krim, Konflikt im Ostteil der Ukraine.

Nach dieser Lesart reiht sich die Eurasische Wirtschaftsunion ergänzend in Moskaus Strategie ein, verloren gegangenen Einfluss in den ehemaligen Sowjetrepubliken wiederherzustellen. So kann die EAWU als Versuch Russlands bewertet werden, eine Alternative zu anderen regionalen Integrationsprojekten, namentlich der Europäischen Union, bzw. seinem Nachbarschaftspolitik für den postsowjetischen Raum, aufzubauen. Zugleich bietet die Wirtschaftsunion Russland die Möglichkeit, die zentralasiatischen Republiken wieder stärker an sich zu binden, die in den letzten Jahren eine Annäherung an China gesucht haben und so den chinesischen Einfluss in der Region zurückzudrängen.

Die vorhandene Literatur zum Themenkomplex russischer Außenpolitik, regionaler Integrationsvorhaben und aktuelle Entwicklungen im postsowjetischen Raum hat bislang jedoch keine Verknüpfung dieser einzelnen Themen geleistet. Neo-realistische Interpretationen russischer Politik beachten bislang nicht den Aspekt, dass eine neo-realistische Strategie, die Macht vorrangig in militärischen Dimensionen definiert, ökonomische Ressourcen zur Verfügung haben muss. So hat die Russische Föderation nun seit der Finanzkrise aus dem Jahr 2007 mit erheblichen konjunkturellen Schwierigkeiten zu kämpfen und weiterhin nicht die Herausforderung gemeistert, seine Wirtschaft in einem Maße zu modernisieren, dass sie auf dem globalisierten Weltmarkt auch unabhängig von Rohstoffexporten konkurrenzfähig ist.



\paragraph{Wirtschaftlicher Einbruch Russlands}
Mag der russische Präsident Vladimir Putin mit der Annexion der Krim-Halbinsel und der vermutlichen Unterstützung der selbstproklamierten Donetzker und Luhansker Volksrepubliken innenpolitisch zwar an Popularität gewonnen haben, so bedeuten beide Krisenherde doch einen erheblichen Abfluss russischer Ressourcen. Sowohl die militärische Hilfe für die 'Separatisten' im Osten der Ukraine wie auch die Modernisierung der maroden Infrastruktur der Krim und die Bezahlung von Sozialleistungen für die dortige russische Bevölkerung, erfordern Mittel, welche für andere Aufgaben nicht verfügbar sind. Parallel wachsen Russlands Rüstungsausgaben, die in einem Bericht für das schwedische Verteidigungsministerium auf umgerechnet 410 Mrd. Euro für den Zeitraum 2011 bis 2020 beziffert werden (vgl. Cooper 2016). 

Angesichts anhaltend niedriger Weltmarktpreise für Russlands wichtigste Exportartikel Erdöl und Erdgas auf dessen Erlöse sich der Staatshaushalt stützt und die Auswirkungen der internationaler Wirtschaftssanktionen in Folge der Krim-Annexion, hat sich die ökonomische Lage für Moskau seit Inkrafttreten der Eurasischen Union Anfang 2015 weiter verschlechtert. Auf das Niveau der Weltmarktpreise für seine Rohstoffe kann Russland praktisch keinen Einfluss nehmen und sieht sich langfristigen Entwicklungen der globalen Energiemärkte gegenüber, die gleich mehrere Herausforderungen für seine Position als Anbieter fossiler Brennstoffe bergen. Die Anstrengungen zu höherer Energieeffizient gekoppelt mit einem stetigen Umbau des Energiemixes hin zu erneuerbaren Energien, nicht nur in den westlichen Industrienationen, sondern auch in China, lässt eine sinkende Nachfrage nach russischem Erdöl und Gas erwarten. Auch haben wiederkehrende Streitigkeiten über Gaslieferungen nach Europa zusammen mit Lieferunterbrechungen die Europäer nach Alternativen zu Erdgas aus Russland suchen lassen, um ihre Versorgungssicherheit zu gewährleisten. In Anbetracht dieser Tendenzen erscheint es fraglich, wie Moskau unter anderem seine Militärausgaben, die sich auf 4.5 \% seines BSP belaufen, auf lange Sicht finanzieren kann, wenn es seine Wirtschaft und Handelsbeziehungen nicht anders als bisher ausrichtet.



\paragraph{Forschungsvorhaben}
Ziel des Forschungsberichts ist keine Neubewertung der Außenpolitik Russlands oder Ursachenforschung für die gewaltsame Eskalation im jüngsten Russland-Ukraine Konflikt, denn zu beiden Themen existiert bereits umfangreiche Literatur. Angesichts der von der russischen Staatsführung gewählten Mittel in den internationalen Beziehung in Kombination mit einer imperialen Rhetorik zur Begründung ihrer Vorgehensweise und Zielsetzung erscheint es wenig erkenntnisversprechend neo-realistische oder auch konstruktivistische Erklärungsansätze erneut zu diskutieren.

Ein neo-realistische Sichtweise russischer Integrationsbemühungen im postsowjetischen Raum lässt die geo-ökonomische Dimension dieser institutionalistisch gekleideten Außenpolitik Moskaus unberücksichtigt. Betrachtet man die wirtschaftliche Entwicklung des Landes seit der Finanzkrise 2007 und stellt dieser den Großmachtsanspruch Moskaus gegenüber, so steht zu beantworten, welche Möglichkeiten der Kreml zur Verfügung hat, seine Ambitionen zu verfolgen. Die bisherige Forschung zu Chancen und Risiken der Eurasischen Union konnte aus Gründen zeitlicher Nähe die ökonomischen Konsequenzen des Russland-Ukraine Konfliktes nicht berücksichtigen. Auf der anderen Seite sind die Auswirkungen der militärischen Auseinandersetzungen in der Ukraine primär unter dem Gesichtspunkt der unmittelbaren Folgen für die russische Wirtschaft untersucht worden, eine Einbettung in den Kontext regionaler Integrationsbemühung lässt auf sich warten.

\paragraph{Methodisches Vorgehen}
Ausgangspunkt des Forschungsberichts ist eine Analyse der wirtschaftlichen Situation Russlands unter dem besonderen Gesichtspunkt, wie sich die Ereignisse der letzten Jahre auf das Staatsbudget ausgewirkt haben sowie welche zukünftige Entwicklungen erwartet werden können. Ausgehend von der Annahme, dass der innen- wie auch außenpolitische Handlungsspielraum des Kremls entscheidend von seinen Budgetrestriktionen bestimmt wird, stellt dies ein messbarer Rahmen für die Analyse russischer Politik gegenüber seinen Nachbarn dar. Hieran soll sich eine Untersuchung des wirtschaftlichen Potentials verstärkter regionaler Integration aus Sicht Moskaus anschließen. Ansätze moderner Handelstheorien bieten hier eine Erweiterung neo-realistischer und konstruktivistischer Perspektiven, welche ökonomische Motive in den internationalen Beziehungen weitestgehend ausblenden. Diese Herangehensweise ist umso erfolgsversprechender als Russland, vor dem Hintergrund seines technologischen Rückstandes gegenüber den westlichen Industrienationen, mit den Handelspartnern der Eurasischen Union bessere Terms-of-Trade erreichen kann, da es hier über mehr komparative Vorteile verfügt.

Für eine Einschätzung der Effektivität der EAWU als regionale Institution liegen mittlerweile mehr empirische Daten vor, als der eingangs genannten Literatur zur Verfügung stand. An dieser Stelle möchte der Forschungsbericht das Augenmerk auf den Wandel der Partnerschaft von der Errichtung eines gemeinsame Außenzolls von 2010 über die interne Zollunion 2012 hin zur jetzigen Wirtschaftsunion legen. Die Leitidee des Forschungsvorhabens wird die Frage sein, inwiefern die Eurasische Wirtschaftsunion für Russland neue wirtschaftliche Möglichkeiten alternativ zu bisherigen Einkommen aus Rohstofferlösen bieten wird. Eine Antwort auf diese Frage lässt auch verglichen mit einer rein ökonomischen Analyse die Effektivität der Wirtschaftssanktionen gegen Moskau besser bewerten. So kann dieser Forschungsbericht einen Beitrag zur Europäischen Außenpolitik leisten, die in der kommenden Zeit mit der völlig neuen Herausforderung des Austritt eines ihrer größten Mitglieder konfrontiert sein wird und zeitgleich hierdurch über geringere institutionelle Kapazitäten verfügen wird.
 



\subsection*{\color{yellowcolor} Vorläufige Literaturliste}


\begin{compactitem}
	
\begin{small}
	
	
	\item [\Rectsteel] \textbf{Antonova, Daria \& Yulia Vymyatina (2014):} Creating A Eurasian Union: Economic Integration Of The Former Soviet Republics, \textsl{Palgrave MacMillan}, Basingstoke, UK.
	
	
	\item [\Rectsteel] \textbf{Blank, Stephen (2014):} The Intellectual Origins Of The Eurasian Union Project, in: \textsl{Starr, S. Frederick \& Svante E. Cornell [Hrsg.]:} Putin's Grand Strategy: The Eurasian Union And Its Discontents, Central Asia-Caucasus Institute, Washington D.C., USA, S.14-28.


	\item [\Rectsteel] \textbf{Cooper, Julian (2016):} Russia's State Armament Programme To 2020: A Quantitative Assessment Of Implementation 2011-2015, \textsl{Bericht für das Schwedische Verteidigungsministerium (Försvars­departemente)}, Stockholm, Schweden.
 

	\item [\Rectsteel] \textbf{Donaldson, Robert H.; Nadkarni, Vidya \& Joseph L. Nodge (2014):} The Foreign Policy Of Russia: Changing Systems, Enduring Interests, \textsl{Routledge}, London, UK.
		
	
	\item [\Rectsteel] \textbf{Dreger, Christian; Fidrmuc, Jarko; Kholodilin, Konstantin \& Dirk Ulbricht (2016):} Between The Hammer And The Anvil: The Impact Of Economic Sanctions And Oil Prices On Russia's Ruble, in: \textsl{Journal of Comparative Economics}, Vol.44 (2), S.295-308.
		
	
	\item [\Rectsteel] \textbf{Feenstra, Robert C. (2014):} Advanced International Trade: Theory and Evidence, \textsl{Princeton University Press}, Princeton, NJ, USA.
	
	
	\item [\Rectsteel] \textbf{Gulam Hassan, Aslam Mohamed \& Kairat Moldashev (2015):} The Eurasian Union: Actor In The Making?, in: \textsl{Journal of International Relations and Development}.
	
	
	\item [\Rectsteel] \textbf{Gvosdev, Nikolas K. \& Christopher Marsh (2014):} Russian Foreign Policy: Interest, Vectors And Sectors, \textsl{SAGE}, London, UK.
	
	
	\item [\Rectsteel] \textbf{Hett, Felix \& Susanne Szkola [Hrsg.] (2014):} Die Eurasische Wirtschaftsunion: Analysen und Perspektiven aus Belarus, Kasachstan und Russland, \textsl{Perspektive}, Friedrich-Ebert Stiftung, Berlin, Bundesrepublik Deutschland.
	
	
	\item [\Rectsteel] \textbf{Jovanovi\'{c}, Miroslav N. (2015):} The Economics Of International Integration, \textsl{Edward Elgar Publishing}, Northampton, MA, USA.
	
	
	\item [\Rectsteel] \textbf{Libman, Alexander \& Vinokurov, Ewgenij (2014):} Do Economic Crisis Impede Or Advance Regional Economic Integration In The Post-Soviet Space?, \textsl{Post-Communist Economies}, Vol.26 (3), S.341-358.
	
	
	\item [\Rectsteel] \textbf{Mankoff, Jeffrey (2013):} Eurasian Integration: The Next Stage, \textsl{Central Asian Policy Brief}, Elliot School of International Affairs, Washington D.C., USA.
	
	
	\item [\Rectsteel] \textbf{Qayum, Faryal \& Yelena Tuzova (2016):} Global Oil Glut And Sanctions: The Impact On Putin's Russia, in: \textsl{Energy Policy}, Vol.90, S.140-151.
	
	
	\item [\Rectsteel] \textbf{Sagorskij, Alexander (2014):} Zwischen Ökonomie und Geopolitik, in: \textsl{Hett, Felix \& Susanne Szkola [Hrsg.]:} Die Eurasische Wirtschaftsunion: Analysen und Perspektiven aus Belarus, Kasachstan und Russland, \textsl{Perspektive}, Friedrich-Ebert Stiftung, Berlin, Bundesrepublik Deutschland, S.3-6.
	
	
	\item [\Rectsteel] \textbf{Sherr, James (2013):} Hard Diplomacy And Soft Coercion: Russia's Influence Abroad, \textsl{Chatham House}, London, UK.
	
	
	\item [\Rectsteel] \textbf{Wisniewska, Iwona (2013):} Eurasian Integration: Russia's Attempt At The Economic Unification Of The Post-Soviet Area, \textsl{OSW Studies}, Centre for Eastern Studies, Warschau, Polen.
	
	
	
	
\end{small}
	
	
	
\end{compactitem}


\subsubsection*{\color{purplecolor} Referenzadressen}

\begin{compactitem}

	\item \textbf{Prof. Dr. Manfred Kerner}\\
					Ihnestraße 20	14195 Berlin\\
					\textsl{Telefon:} +49 (0)30 838 52090\\
					\textsl{Email:} kerner.steglitz@t-online.de 
	
	
	\item \textbf{Prof. Dr. Miranda Schreurs}\\
					Ihnestraße 22	14195 Berlin\\
					\textsl{Telefon:} +49 (0)30 838 56654\\
					\textsl{Email:} Miranda.Schreurs@fu-berlin.de

\end{compactitem}







\end{document}



